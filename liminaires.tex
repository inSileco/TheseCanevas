%----------------------------------------------------------------------%
% Liminaires de la thèse.                                              %
% UQAR septembre 2013                                                  %
% ---------------------------------------------------------------------%

% ----------------------------------------------------------------------%
% 1- Page titre.                                                        %
% ----------------------------------------------------------------------%

\Pagetitre
\cleardoublepage
% ----------------------------------------------------------------------%
% inclusions qui pourraient mériter d'être incluses dans le .cls
% (commentez si non-nécessaire)
% 1.1 - Composition du Jury.                                           %
\thispagestyle{empty}

\null
\vfill
\noindent \textbf{Composition du jury:}\\
\vspace{1cm}

\begin{singlespace}
  \noindent \textbf{QQ1, président du jury, Université du Québec à Montréal}\\

  \noindent \textbf{Dominique Gravel, directeur de recherche, Université du Québec à Rimouski}\\

  \noindent \textbf{Phillipe Archambault, directeur de recherche, Université de Montpellier}\\

  \noindent \textbf{QQ1, examinateur interne, QQpart}\\

  \noindent \textbf{Q1, examinateur interne, Université du Québec à Rimouski}\\

  \noindent \textbf{QQ1, examinateur externe, QQpart}\\
\end{singlespace}

\vspace{2cm}
\noindent Dépôt initial le 12 septembre 2017
\hspace{3cm}
Dépôt final le [-]


\cleardoublepage

% % 1.2 - Avertissement biblio.
\thispagestyle{empty}

\vspace{2cm}
\begin{center}
UNIVERSITÉ DU QUÉBEC À RIMOUSKI\\
Service de la bibliothèque
\end{center}

\vspace{3cm}
\begin{center}
Avertissement
\end{center}


\vspace{1cm}

\noindent La diffusion de ce mémoire ou de cette thèse se fait dans le respect des droits de son auteur, qui a signé le formulaire {\itshape \og Autorisation de reproduire et de diffuser un rapport, un mémoire ou une thèse \fg}. 
En signant ce formulaire, l’auteur concède à l’Université du Québec à Rimouski une licence non exclusive d’utilisation et de publication de la totalité ou d’une partie importante de son travail de recherche pour des fins pédagogiques et non commerciales. 
Plus précisément, l’auteur autorise l’Université du Québec à Rimouski à reproduire, diffuser, prêter, distribuer ou vendre des copies de son travail de recherche à des fins non commerciales sur quelque support que ce soit, y compris l’Internet. 
Cette licence et cette autorisation n’entraînent pas une renonciation de la part de l’auteur à ses droits moraux ni à ses droits de propriété intellectuelle. 
Sauf entente contraire, l’auteur conserve la liberté de diffuser et de commercialiser ou non ce travail dont il possède un exemplaire.



\cleardoublepage
% % 1.3 - Dedicace.
\thispagestyle{empty}

\begin{minipage}[l]{0.45\textwidth}

\end{minipage}%
\hfill
\begin{minipage}[r]{0.5\textwidth}
\begin{quotation}
\begin{doublespace}

à qui tu veux,

\end{doublespace}
\end{quotation}
\end{minipage}%

\cleardoublepage

% ----------------------------------------------------------------------%


% ----------------------------------------------------------------------%
% 2- Remerciements.                                                    %
% ----------------------------------------------------------------------%

\remerciements
\selectlanguage{french}
\input{limi_src/cimer.tex}

% [Cette page est facultative; l’éliminer si elle n’est pas utilisée. Les remerciements peuvent aussi être intégrés à l'avant-propos. C’est dans cette section que l’on remercie les personnes qui ont contribué au projet, les organismes ou les entreprises subventionnaires qui ont soutenu financièrement le projet.]



% ----------------------------------------------------------------------%
% 3- Avant-propos.                                                     %
% ----------------------------------------------------------------------%

\avantpropos
\selectlanguage{french}
La j'ai fait des merci formels!



% [Cette page est facultative; l’éliminer si elle n’est pas utilisée. L’avant-propos ne doit pas être confondu avec l'introduction. Il n’est pas d’ordre scientifique alors que l’introduction l’est. Il s’agit d'un discours préliminaire qui permet notamment à l'auteur d'exposer les raisons qui l'ont amené à étudier le sujet choisi, le but qu'il veut atteindre, ainsi que les possibilités et les limites de son travail. On peut inclure les remerciements à la fin de ce texte au lieu de les présenter sur une page distincte.]



% ----------------------------------------------------------------------%
% 4- Resume/Abstract                                                           %
% ----------------------------------------------------------------------%

\resume
\begin{singlespace}
La biogéographie est l'étude des mécanismes et des processus qui
influencent la répartition géographique des être vivants.

\begin{quote}
Mots clés: Biogéographie, interactions biotiques, réseaux écologiques,
contraintes abiotiques, co-occurrence, théorie de la biogéographie des
îles, théorie métabolique de l'écologie.
\end{quote}

  % [Le résumé en français doit présenter en 350 mots maximum pour un mémoire et en 700 mots pour une thèse : (1) le but de la recherche, (2) les sujets étudiés, (3) les hypothèses de travail et la méthode utilisée, (4) les principaux résultats et (5) les conclusions de l'étude ou de la recherche.]

\end{singlespace}
\cleardoublepage


\abstract
\begin{singlespace}
Biogeography is the study of the mechanisms and processes that control
the geographical distribution of plants and animals.

\begin{quote}
Keywords: Biogeography, biotic interactions, ecological networks,
abiotic constrains, co-occurrence, theory of island biogeography,
metabolic theory of ecology.
\end{quote}


  % [L'abstract doit être une traduction anglaise fidèle et grammaticalement correcte du résumé en français.]

\end{singlespace}
\cleardoublepage




% ----------------------------------------------------------------------%
% 5- Table des matières.                                               %
% ----------------------------------------------------------------------%

\tabledesmatieres



% ----------------------------------------------------------------------%
% 6- Liste des tableaux.                                               %
% ----------------------------------------------------------------------%

\listedestableaux

% ----------------------------------------------------------------------%
% 7- Table des matières.                                               %
% ----------------------------------------------------------------------%

\listedesfigures

% ----------------------------------------------------------------------%
% 8- Liste des abréviations (optionnel).                               %
% ----------------------------------------------------------------------%

\listeabrev
\begin{liste}

\item[DOI]~: \textit{Digital Object Identifier}; identifiant numérique d'objet.

\item[GIEC]~: Groupe d'experts Intergouvernemental sur l'Évolution du Climat.

\item[IPBES]~: \textit{Intergovernmental Science-Policy Platform on Biodiversity and Ecosystem Services}; Plateforme intergouvernementale sur la biodiversité et les services écosystémiques.

\end{liste}



% ----------------------------------------------------------------------%
% 9- Liste des symboles (optionnel).                                   %
% ----------------------------------------------------------------------%

% \listesymboles
% \begin{liste}
% \item[SYMBOLE 1] Ceci est la définition du symbole 1.
%
% \item[SYMBOLE 2] Ceci est la définition du symbole 2.
%
% \item[SYMBOLE 3] Ceci est la définition du symbole 3.
% \end{liste}

% ----------------------------------------------------------------------%
% Fin des liminaires.                                                  %
% ----------------------------------------------------------------------%

\cleardoublepage
